% study_notes.tex
\documentclass{article}
\usepackage[utf8]{inputenc}
\usepackage{amsmath}
\usepackage{graphicx}
\usepackage{hyperref}
\usepackage{geometry}
\usepackage{cite}
% Basic packages
\usepackage[utf8]{inputenc}
\usepackage[T1]{fontenc}
\usepackage[english]{babel}
\usepackage{lmodern}

\usepackage{color}
\usepackage{graphicx}
\usepackage{subcaption}


\geometry{a4paper, margin=1in}

\title{Study Notes for AWR Tutorial}
\author{Your Name}
\date{\today}

\begin{document}

\maketitle

\begin{abstract}
These are study notes for the Advanced Writing and Research (AWR) tutorial.
\end{abstract}

\tableofcontents

\section{Introduction}
This is an introductory tutorial to use AWR from the Roland device.

\section{Videos}

\subsection{Design Troubleshooting for Stability Part 1}
Instability arises from the use of feedback and gain together. This can be controlled by adding loss or adjusting bypassing in the lab to stabilize the circuit.

\textbf{From AWR DE Tutorial 5}
\begin{itemize}
    \item After selecting the new schematic, name it and then press \texttt{Ctrl + L} to add elements.
    \item To see the details of a component, right-click and select \texttt{Help}.\cite{paper_2}
    \item You can add equations as in QUCS and use the factors in the component details.
    \item To tune an item or a factor on the graph, click the \texttt{Tune} tool above and then click on the factor; it will change to a blue color.
    \item Set the limit of the chosen factor and then click the tuner. It is behind the identifier tool that chose the factor to tune. You will see the variable tuner has limits and then you can change it above and below.
    \item To plot the output voltage (\( V_{out} \)), you need a voltmeter. Connect it in shunt with the load resistor \( R \) chosen from the element menu.
    \item To measure the DC characteristics of the amplifier (\( V_{out} \) vs \( V_{in} \)), you need a linear graph. Add a new measurement, select \texttt{Nonlinear} and then \texttt{Voltage}. Choose \( V_{DC} \) on the measurement component and place the voltmeter on the vertical axis. For the x-axis, use the voltage from the source.
    \item Click \texttt{Apply} and then choose the simulation tool on the schematic.
\end{itemize}

\textbf{From AWR DE Tutorial 6}
\begin{itemize}
    \item From the elements menu, you can also find the voltage source (V source). If it is AC, to measure it you will need a voltmeter similar to the one used on the load resistor.
    \item The working frequency is a global value defined for the overall project from the project options menu. Define the frequency range and then from the global units icon above, choose the frequency and change its unit to kHz in this case.
    \item The netlist file that can be imported into AWR has the extension \texttt{.cir}. From the netlist menu, select \texttt{Netlists} and then choose \texttt{Import Netlist}.
    \item With \texttt{Ctrl + K}, you can add the important netlist.
    \item After adding the symbols, you can change them from square and normal device shapes from \texttt{Properties > Symbols}, then choose the needed shape.
    \item To add the voltage source, add \texttt{DCVS} and \texttt{ACVS} from the elements.
\end{itemize}

\textbf{From AWR DE Tutorial 7}
\begin{itemize}
    \item The models that can be used in AWR are the PSpice models with \texttt{.cir} extension.
    \item These are called AWR netlist files.
    \item From \texttt{Steer 03.Wideband Amplifier Design Using the Negative Image Model, Part F}, we can see that the model used with the transistor is added in the data files above the graphs.
\end{itemize}

\textbf{From Impedance Matching AWR}
\begin{itemize}
    \item From project options, you can change many of the item units.
    \item You can use the cookbook for ADS and check the LNA simulation part and try to replicate it in AWR.
    \item To measure the input impedance, it is a linear measure, and you can find it on the graph after adding a new measure. Choose \( Z_{in} \).
    \item Define \( Z_{in} \) unit, the port number, and the source name. If there is a transmission line after the input port, it will appear.
    \item It is important to sweep the frequency and define the number of points for measurement to determine at which frequencies the device should read a measurement. This can be done from the project options window.
    \item To add another plot on the graph, click \texttt{Duplicate Measurement} and then from the window called \texttt{Modify Measurement}, you can add the changes between the two lines.
    \item For the 50-ohm resistor, if you change the \( Z_{in} \) measure from 75 ohms with the transmission line to 50 ohms, you will see matching at 50 ohms for \( Z_{in} \) real and the imaginary value is zero, indicating matching.
    \item In the case of 75 ohms with a 50-ohm transmission line, you will see the imaginary part increase with frequency from -20 to +20. If the imaginary part is greater than 0, it means positive reactance, indicating inductance, and negative reactance means it is capacitive.
    \item If you remove the resistor and use just the transmission line and simulate, you will see the real and imaginary parts are 0 ohms at 2.4 GHz if the transmission line is open, not grounded. If grounded, there will be a very large \( Z_{in} \) at 2.4 GHz.
    \item In the matching case, measure the impedance on the Smith chart and check where the point is for your frequency. Connect a series capacitor and a parallel inductor if the point on the Smith chart is on the capacitive part under the middle line. Then tune the values of C and L and check which one has a larger effect on the point. Tune it to get the point above on the inductive part, then with the capacitor value change it to reach the middle line indicating the point is totally matched.
    \item After adding a microstrip line with T-line used as a load, it should be matched when the length and the width give a 50-ohm for the material used on the substrate. In this case, the microstrip line and the T-line are 50-ohm, meaning they should be matched around the frequency of the T-line, which is 2.4 GHz. Matching can be seen from the impedance linear graph or from the Smith chart.
    \item If you have just one port, you will measure on Smith just \( S_{11} \). When measuring, it is a must to use the MSUB on the schematic to define the substrate for the transmission line. The values of the variables on the MSUB are taken from the window where we define the width and the length on the MSline, meaning \( \epsilon_r \) and tan \( \delta \) and so on.
    \item When matching with MSline, define the length and then you can add a parallel one in front of the MSline and define the length and tune it to get the matched point. Adding the admittance might help on Smith to get a line to move with it. On the new line, you can play with the width and the length.
    \item The microstrip TEE or MTEE is used to connect the MS lines instead of connecting them together, which might lead to parasitic capacitance. So the idea is to make the design and see the Smith chart and match with the MS lines, and then control the length and width of the MS lines to get the matched point. Then connect them with the MTEE with a dollar sign which will connect the three sides with the equivalent width for the line.
    \item If an MS line is not grounded, you need to connect the Mopen line, which is an MS line with open termination.
    \item To see the length of the MS lines, you can see the layout graph and then import the other component models that define the layout. AWR requires some experience because it does not always have the best documentation.
\end{itemize}

\textbf{Microstrip Antenna AWR Lab Video}
\begin{itemize} 
    \item This video is important for working with T-lines.
\end{itemize}

\section{Technical Note from the CEL Blog}
The CE3512K2 is an air cavity package structure which is effective in reducing RF power loss consumed by the packaging material. For HEMT, the gate has high impedance at DC, so there is only a need for bias at the drain and connecting \( V_g \) directly. The bias at the drain is sufficient to use just the inductor, and it should be high enough to provide good RF isolation at the operating frequency. The inductor can also be part of the matching network. A shunt capacitor provides a low impedance point, creating a low pass filter for noise and unwanted signals from the DC source. As frequency increases, parasitic effects become dominant, necessitating transmission line-based components in both bias and matching circuits.

\textbf{Transmission Line Matching}
The quarter wavelength (\( \lambda /4 \)) transmission line is a crucial element in RF design.
the (\(\lambda/4 \)) can replace he inductor on rf isolation. and for this technique there is no frequency limit.
Radial and rectangle open stub are the most common Tlines on rf.
thr coupling capacitors has to undesirable effects it has high insertion loss as a fraction of dB directly degrades the
noise figure by the same amount and the matching network is become complicated due to the parasitic effect of the high 
frequency capacitors.

output series capacitor has negatable impact on the circuit performance 

you will need bias tee if you you wanna exclude using the input capacitors to prevent dc coupling with the instrument during the measurement.
this only on noise figure measurement. 

Tlines component on the matching network usually used on the cas for the freq higher than  10Ghz

Ampleon is a good company for rf transistor with application note and references.
as stated on the website  edaboard.com   after you have the transistor model you can chose a bias point then stability analysis 
with Rollet factor then design the input and output matching. also the power transistor has optimum load impedance
that maximize the delivered power or the gain.

one simple measurement is to make representation for the S param and then compare between QUCS and AWR 


 \textbf for Simulation 

 add S2p on the data files by import then add it on the schematic as rectangle then click and from Properties change the symbol.
 then follow the circuit stubs as on the Matching with the AWR video.
 the bias circuit is to design the bias point and the matching to work on  50ohm 
 on HEMT  transistor has highrt breakdown voltage and high operational temp.
$
\Gamma (S) = \Gamma (opt) $  must be first for LNA
connect the bias circuit and define the bias voltaged and calculate Rg and Rd and the capacitor and the inductors






\section{From github (https://github.com/aliruveycan/Maximum-Earnings-using-AWR-Microwave-Office-Amplifier-Design-and-Simulation.git)}

to simulate the  transistor on the data files upload the s parameter file 
and on the graphs plot all S parameters graphs.

then you can match with the Tlines  length as on the AWR 1 hour tutorial for Matching.








\section{Diffrent pdf notes from google Image}
\textbf{From Ultra wid band amp by chengcheng Xu}
Rf amplifier is used to amplify small signal amplitude to a large amplitude  in transmission and receive signal from antennas.


if the BW from 1.5 to 3.5  Ghz it means that  the amplifier can  be called 2 GHz amp so our amplifer can be called 
1GHz  or 800 MHz  or 600MHz   amplifer for the SNSPDs 


 Not understood   phrase is the final gain is 20 dB with the power gain slightly below 30 dBm.

 if the transistor IV curve is mostly linear then the output signal gain will be linear otherwise gain will be non linear.

 Class A amplifier has poor efficiency when converting the dc power to high freq output power 
 the lost power will br converted to heat on the device this power that is not converted to rf power 
but it provide great output to input linearity 
#

Class B more efficient than A but it scarifies the output linearity to  input signal ratio. lead to distortion on the output
signal half of the signal only amplified 

push pull configuration with two transistor may solve the class B amplifier issue.

if the spice model is not provided so the setup a bias network is not important you will use on the s2p file form the 
provider then you need to match the output of the transistor to the output matching circuit 
also consider the stability of the amplifier if it is conditional on unconditional stable.

the S22 matching with the output matching network is created first than the inpupt matching network 

if S11 is not matched all the input power will not be deliver to the amplifier 

the biasing circuit or bias point use one that is recommended from the the provider bias points means the points for the 
s2p files that provided Vds and Vgs and Id  

chose the bias point that has gain near to your interested freq on the thesis they use s2p at 3v 60 mA bias this can be show from the graph for 
MSG/MAG and S21 dB  you see that S21 has gain at the needed frequency.


this bias point also avilable on the AWR software so no need to import the file on the Microwave office software 


P1dB  compression point is also important  as the transistor it self is not linear so the output / input ratio can't be linear
 
it is defined as the point where the device output power and the linear model output different
by 1dB.

this point where the device also start to provide Nonlinear behavior so you look at the p1dB point on the data sheet 

if you wanna use another transistor you should follow the same process.

the FPD2250 has 1dB point 31dBm  and if the gain is 10dB  so the max power input to the transistor
is 21dBm so you need to see the conversion of the power values to dBm and check the P1dB effect. the 21dBm will be 
above the output of the 1 st stage.

the stability is not related to which gain value of BW is the device it self is unstable  this will not has an effect
on what gain or other parameter so we need to use stable network to prevent oscillating output


the stability on Microwave office is tested by K() and B1 parameter and also by SCIR1() and SCIR2() parameters of the device

K>1 and B not < 0 

SCRR 1 will create solid line  and dashed lines circles the inside or out side the l´solid line circle 

the  solid circle will be the stable one. the goal is to be inside the stable region 

\textbf{ Design steps}
\begin{itemize}
    \item add the transistor model and change it to symbol or leave it as it is.
    \item check also if it is avilable on AWR or not at the beaning
    \item design a bias point circuit should be stable or should be called stability circuit. ther is no voltage sources as on fig on the tutorial just Rd and Rout and Rg and the two ports also you  might need to use MSline and control it's length tune the component then see the stability for K and B values.
    \item the above  step is  reoeated for the two the stages on thier own.
    \item this stability test can be done on smith plot with the circles need te studied alone.
\end{itemize}

\textbf{Wideband Matching}
\begin{itemize}
    \item the goal is to has a Wideband at the first gain is not important at this point as the BW.
    \item so Gmax is not needed now check table one on the tutorial pdf
    \item matching method are Chebyshev matching  and lumped elements matching
    \item Chebyshev matching is based on Chebyshev polynomial and the matching network is mostly consist of quarter wave transformer
    and from the BW and quarter wave transformer section number the impedance of each quarter wave transistor is generated
    Chebyshev Matching is on figure 12
    \item  then you need to see the matching on smith see S11 and S22 it should be nearly perfect match 
    means it is circulated around the center point smith plot.
    \item  as shown on figure 12 Chebyshev has 6 Tline for each S11 and S22 and the and the length of the quarter wave transistor
    need to be quarter of the signal wavelength. so the reault now for Rf application is not realestic and the
    and the design will be very large and amplifier detentions will be large
    \item so the Chebyshev matching is not implemented  here.
\end{itemize}



\textbf{lumped element match}
\begin{itemize}
    \item 
\end{itemize}


\section{matching}

on smith for any point to match it move with circle until you get a point on a circle that pass by the smith center 
which is 50 ohm matched.

on impedance smith series R go go right toward the center line 
series C goes down
series L goes up


on the admittance plane the shunt R,L,C  goes the same L up 
C down but R left.

\textbf{on Lec 7 impedance match with smith video is very nice}

if $$z_o = 50 ohm and Z_s = 50 ohm so Z_i the normalized impedance is 1 means full matched$$
$Z_i$ which is represented on smith chart 

adding resistor will add loss to your design so using L and C is is enough and better.
impedance match with the Tlines is very diffcult compared to Lumped element but the idea is the 
same there is numerating on smith for the matching lines.
bit the middle line for the real is the same but the imagnary numbers are written now interms of \Lambda.




\textbf{Txline  feature from AER Videos}

from tools chose txline it is used to know how much width should the line be to be $50 ohm$
or how length should be to 90 deg at 80 Ghz for example
on the txline window change to mm on length and width 
using the arrow on the middle to change between the calcuation then use the length and width on thes chematic then simulate to see 
the smith chart  you will see the point is on the middle if smitt


\textbf{Sweep feature}
on the parameter you want sweep click right then chose sweep  define a start and end and steps.
 and define also by equation a value to initialize the variable.
\textbf{Reflection coefficient measurement}
from the graph and S11 measure but at the beginning you have to add measure port at the input 
all of that is on the polar plot not smith
then tune the Tline length.
and see the effect on S11 

if you change the port to harmonic balance port you will be able to see the 
I V values on this device. time waveforms  of I V.
on elements it is called port1
you will see the wave source.after it you need a V meter or I meter 
from element chose V_meter and also I meter 
I meter on series after port and v meter on shunt.
from the graph rectangle then Nonlinear graphs then voltage then vtime.

on plot chose properties then measurement then  chose which graph on left and which on right. side.

on the Tline is length 0  the I V time relation are on phase when change the length the VSWR is the same and the 
real part of the reflection coefficient is the same  but hte phase deg changes always
this means the phase difference between IVtime relation will change.

due to the phase changes the Tline is considered as network from inductors series and capacitor parallel
with the the voltage source.

VSWr is the ration between $$V_max$$ / $$V_min$$





\textbf{there are some simulation AWR videos on the Microwave labcast channel}


\textbf{Simulation} 
\begin{itemize}
    
    \item import the sparmeter and make graph for the device at the beganing before anything.
    \item then you can make your graph from the graph tool
    \item first stabilize your circuit connet in out series R and shunt R at the out this will leads to uncoditional stability.
    \item then you can check that by the by K measure or SCIR1 ans SCIR2 for stability circle.
    \item now need to match over the wide band this is most important than getting gain. so BW is the important now
    \item the Match step is the most difficult this can be by the TLines and the lumped elements 
    \item most popular on AWR the Tline method from the Txline tool you can characterize your Tline properties.
    \item Match output before the input it is easy and important for the gain means match s22 of the transistor with the output load or matching.
    \item then Match the input S11 for the transistor with the input circuit to be sure that the power will be deceived to the transistor otherwise it will not.
    \item now Design  input match and output match sepertly and check them on smith the trick here is to make the whore freq range around the match point middle of smith
    \item and define only the operating or project frequencies on fron Dc to 1 ghz and define for th txline frequencies the middle point 500 MHz or less it means to make the Wideband match you need to match around the middle frequency.
    \item after you make the matching circuits make blocks and connect them to the in and out of the transistor 
    \item then check the gain now.
    \item then change the port to source nd simulate the power on the device to see the total power output on this case you will see only the part where the S parametr measured mean the linear region not the whole characterstcis this due to no Nonlinear model avilable for the transistor
    \item and due to no Nonlinear model the Nonlinearity  mean P1dB and OPI3 points of the device will be read only from the datasheet as mentioned on the pdf tutorial by chengcheng zu.
    on the Tlines from the txline which is considered to be real Tline insted of the ideal one.
    you will use the stability T junctions  when connect 3 Diffrent Txlines.
\end{itemize}


\textbf{the matching circuit notes}
\begin{itemize}
    \textbf{from AWR mtch tutorial}
    \item ideal Tline is lossles gives perfect phase delay on a spesific frequency and it like on class
    \item under transmission lines you will fined phase menu then chose Tlin ans Tloc are on the  same menu. these are uesd for the lossles case.
    \item For the Tline Z0 is the impedence keep it $50 ohm$  and EL is the  electrical length it not the physical length it is the phase delay in degrees that will apply to the signal F0 is the center frequency is very important with EL .
    \item to simulate the input impednce on Ohms from the linear curves chose zin and only magnitude then make your graph and measure also  from Port 1
    \item real Zin is the the ohm value  and the img Zin graph show the reactance whin define the capacitance and the inductance if it is negative or positive. this where the matching starts.
    \item if you mesure the Zin for the Tline with GNd only it will show at the center freq the a huge value at a single freq whic is define an open circuit due to high impedence value.
    \item if you wanna control a freq on smith juss chose it as the central freq on the Tlines then dee it is point on smith then move it up and sown based on the admittance and impedance plots .  and using resistors not prefiered as it introduce losses .
    \item when start match with the lumped elements always start with the L netweork of L and C together and then control it notice the higher the capacitance the lower the impedence as $Z=j/\omega C$ 
    \item 
    
\end{itemize}









With the dielectric function $\epsilon$, which describes the response of the material to external electromagnetic fields. $\epsilon = \epsilon_1 + i \epsilon_2$ is a complex function, which will in general depend strongly on the frequency of $\Vec{E}$ and also on the direction, if the material is anisotropic. The sometimes more intuitive complex refractive index $\Tilde{n}$ can be gained from $\epsilon$ throug \cite{paper_2}:

\begin{equation}
    \sqrt{\epsilon} = \Tilde{n} = n + i\kappa
\end{equation}














\subsection{Key Paper 1}
Discuss the first key paper. \cite{author2021paper}

\subsection{Key Paper 2}
Discuss the second key paper. \cite{author2022paper}



\section{Conclusion}
Summarize the key points discussed in the tutorial and their implications.

\bibliographystyle{plain}
\bibliography{references}

\end{document}
